%% 
%% Copyright 2007-2019 Elsevier Ltd
%% 
%% This file is part of the 'Elsarticle Bundle'.
%% ---------------------------------------------
%% 
%% It may be distributed under the conditions of the LaTeX Project Public
%% License, either version 1.2 of this license or (at your option) any
%% later version.  The latest version of this license is in
%%    http://www.latex-project.org/lppl.txt
%% and version 1.2 or later is part of all distributions of LaTeX
%% version 1999/12/01 or later.
%% 
%% The list of all files belonging to the 'Elsarticle Bundle' is
%% given in the file `manifest.txt'.
%% 
%% Template article for Elsevier's document class `elsarticle'
%% with harvard style bibliographic references

%%\documentclass[preprint,12pt]{elsarticle}

%% Use the option review to obtain double line spacing
%% \documentclass[preprint,review,12pt]{elsarticle}

%% Use the options 1p,twocolumn; 3p; 3p,twocolumn; 5p; or 5p,twocolumn
%% for a journal layout:
\documentclass[final,1p,times]{elsarticle}
%% \documentclass[final,1p,times,twocolumn]{elsarticle}
%% \documentclass[final,3p,times]{elsarticle}
%% \documentclass[final,3p,times,twocolumn]{elsarticle}
%% \documentclass[final,5p,times]{elsarticle}
%% \documentclass[final,5p,times,twocolumn]{elsarticle}

%% For including figures, graphicx.sty has been loaded in
%% elsarticle.cls. If you prefer to use the old commands
%% please give \usepackage{epsfig}

%% The amssymb package provides various useful mathematical symbols
\usepackage{amssymb}
\usepackage{amsmath}
\usepackage{hyperref}

%% The amsthm package provides extended theorem environments
%% \usepackage{amsthm}

%% The lineno packages adds line numbers. Start line numbering with
%% \begin{linenumbers}, end it with \end{linenumbers}. Or switch it on
%% for the whole article with \linenumbers.
%% \usepackage{lineno}

\journal{mextractmodel documentation}

\begin{document}

\begin{frontmatter}

%% Title, authors and addresses

%% use the tnoteref command within \title for footnotes;
%% use the tnotetext command for theassociated footnote;
%% use the fnref command within \author or \address for footnotes;
%% use the fntext command for theassociated footnote;
%% use the corref command within \author for corresponding author footnotes;
%% use the cortext command for theassociated footnote;
%% use the ead command for the email address,
%% and the form \ead[url] for the home page:
%% \title{Title\tnoteref{label1}}
%% \tnotetext[label1]{}
%% \author{Name\corref{cor1}\fnref{label2}}
%% \ead{email address}
%% \ead[url]{home page}
%% \fntext[label2]{}
%% \cortext[cor1]{}
%% \address{Address\fnref{label3}}
%% \fntext[label3]{}

\title{Mathematical modeling of micro-extraction mechanism
}

%% use optional labels to link authors explicitly to addresses:
%% \author[label1,label2]{}
%% \address[label1]{}
%% \address[label2]{}

\author{Pedro S. Peixoto}

\address{University of São Paulo}

\begin{abstract}
%% Text of abstract
Mathematical modeling of micro-extraction mechanism. Adapted for Gas-Diffusion or Solid-Phase.
\end{abstract}

%%Graphical abstract
%\begin{graphicalabstract}
%\includegraphics{grabs}
%\end{graphicalabstract}

%%Research highlights
%\begin{highlights}
%\item Research highlight 1
%\item Research highlight 2
%\end{highlights}

\begin{keyword}
%% keywords here, in the form: keyword \sep keyword

%% PACS codes here, in the form: \PACS code \sep code

%% MSC codes here, in the form: \MSC code \sep code
%% or \MSC[2008] code \sep code (2000 is the default)

\end{keyword}

\end{frontmatter}

%% \linenumbers

%% main text
\section{Mathematical Model and Numerical Method}
\label{sec:model}

In this section we will describe a slight modification of the diffusion model proposed by \citet{zhang1993headspace} in order to build a numerical model that is adequate for variable discontinuous diffusion processes, such as those encountered in micro-extraction techniques. We first explain in details a general diffusion model, that is actually very well known in the literature (see details \citet{leveque1992numerical}).

We will assume the full device in which the diffusion process will happen has two uniform dimensions, say $y$ and $z$, and one dimension ($x$) for which we have compartments of different sizes. 

The main assumption made about the process is that the dominant/relevant diffusion process occurs only in one direction ($x$), therefore being uniform in the other dimensions ($y$, $z$). As a result, we will obtain a model that is independent of the device size in $y$ and $z$ direction.

\subsection{Compartmental model}

Consider a single compartment of the device to have dimensions $(L_x, L_y, L_z)$. Fick's law assumes that the diffusion flux ($J$), providing the mass per area unit and per time unit, crossing a section area at a certain position in the x-coordinate and time $t$, is given by
\begin{equation}
J(x,t)=-D(x) \frac{\partial C(x,t)}{\partial x},
\end{equation}
where $C(x,t)$ denotes the concentration of the material (amount of substance per volume) undergoing a diffusion process, at a certain place and time; $D(x)$ denotes the diffusion coefficient (cross section area per time units), which depends on the environment that contains the substance, therefore may be different in different $x$ positions.

We will subdivide the compartment into a set of discrete points. If we define the compartment to be positioned in the interval $[0, L_x]$, we can subdivide it into $N$ sub-intervals of size $dx$, such that $x_i=i\,dx$, $i=0, 1, ..., n$ with $dx=L_x/n$. The mass for each control volume may be defined as 
\begin{equation}
M_{i}(t)=L_y\, L_z\,\int_{x_{i-1/2}}^{x_{i+1/2}}C(x,t) \,dx,
\end{equation}
on a sub-interval $[x_{i-1/2}, x_{i+1/2}]$, where $x_{i+1/2}$ indicates the midpoint of the interval $[x_i, x_{i+1}]$.



The mass change in time depends only on the lateral fluxes of the control volume, therefore,
\begin{equation}
\frac{d M_{i}(t)}{dt}= L_y\, L_z (J(x_{i-1/2}, t)-J(x_{i+1/2})).
\label{eq:localmassconserv}
\end{equation}
Note that the flux in the position $x_{i-1/2}$ has positive sign, as it is indicating inward flow into the compartment, and the flux in the position $x_{i+1/2}$ has negative sign, since the flow at this point indicates outward flow.


Equation \eqref{eq:localmassconserv} may be re-written, assuming sufficient regularity of the functions, as
\begin{equation}
\int_{x_{i-1/2}}^{x_{i+1/2}} \frac{\partial C(x,t) }{\partial t}  \,dx =  -\int_{x_{i-1/2}}^{x_{i+1/2}}\frac{\partial J(x,t)}{\partial x} 
\end{equation}
where we note that the problem is therefore independent of $L_y$ and $L_z$. Also,  in order to be valid for any interval of the form $[x_{i-1/2}, x_{i+1/2}]$, it requires that
\begin{equation}
\frac{\partial C(x,t) }{\partial t}=\frac{\partial  }{\partial x}
\left(D(x) \frac{\partial C(x,t)}{\partial x}\right), \quad x\in (0, L_x).
\label{eq:dif_model}
\end{equation}
If $D(x)$ is independent of $x$, we arrive at the same familiar equation used in \citet{zhang1993headspace}. However, we will assume the media is anisotropic, so that even when the diffusivity is not constant we will have mass conservation. By using the model proposed in \citet{zhang1993headspace}, compactly written as $C_t=DC_{xx}$, we can't, in general, expect local mass conservation of the system. 

At the boundaries of the compartment ($x_0=0$ and $x_n=L_x$) we may enforce one of two boundary conditions of interest:
\begin{enumerate}
\item[(i)] A flux condition, defining an exact value of $J(x_0, t)$ or $J(x_n, t)$ for all times, also known as Neumann boundary condition. The interesting value for our problem is to set a no flux boundary conditions, which may be imposed by simply forcing $J(x_0, t)=0$ or $J(x_n, t)=0$, or, equivalently, that
\begin{equation}
\frac{\partial C(x_0,t)}{\partial x}=0, \quad \frac{\partial C(x_N,t)}{\partial x}=0.
\end{equation}
This condition is useful for if the compartment if the first ($J(x_0, t)=0$) or last compartment ($J(x_n, t)=0$), so that the system is closed (conserves mass globally).
\item[(ii)] An interface boundary condition, defined when the boundary meets another compartment. At these interfaces we will impose that the concentrations always respect a certain proportion ratio, that is, the limits from left and right of the transition point are such that, for example in case of an interface at point $x_n$, 
\begin{equation}
\lim_{x\rightarrow x_n^{-}} C(x, t) = K \lim_{x\rightarrow x_n^{+}} C'(x, t),
\end{equation}
where $K$ defines the proportionality gap and $C'$ is the concentration on the neighbor compartment.
Additionally, the interface boundary conditions require that $J(x_0^-, t)=J(x_0^+, t)$ or  $J(x_n^-, t)=J'(x_n^+, t)$, where $J'$ is the flux on the neighbor compartment, which mean, for example for the point $x_n$,
\begin{equation}
\lim_{x\rightarrow x_n^-}D(x) \frac{\partial C(x,t)}{\partial x} = \lim_{x\rightarrow x_n^+} D'(x) \frac{\partial C'(x,t)}{\partial x},
\end{equation}
which basically states that, in the limit, the outward flow of the compartment ($x_n^{-}$) matches the inward flow in the adjacent compartment ($x_n^{+}$), ensuring mass conservation.

\end{enumerate}

Consider two compartments side by side, with no flux on the outer boundaries and an interface condition in between the compartments:
\begin{itemize}
\item Is this continuous problem well posed for any $K$, $D_1$, $D_2$ and given initial conditions? 
\item We know a steady state solution exists for this problem if the initial condition is constant on each side respecting the gap condition. But does the system allow existence of transient solution? Are they unique given the parameters?

\end{itemize}

\section{Numerical Model}

Without loss of generality, we may assume $L_y=1$ and $L_z=1$, as the solution of the problem is independent on these values. Now, the mass variation at an interval $[x_{i-1/2}, x_{i+1/2}]$ can be written as 
\begin{equation}
\frac{d M_{i}(t)}{dt}= J(x_{i-1/2}, t)-J(x_{i+1/2}).
\end{equation}
The diffusion fluxes may be obtained via centered differences (second order accurate with respect to $dx$) formula as,
 \begin{equation}
 J(x_{i-1/2}, t)=-D(x_{i-1/2})\frac{C_{i}(t)-C_{i-1}(t)}{dx}+O(dx^2),
 \end{equation}
  \begin{equation}
 J(x_{i+1/2}, t)=-D(x_{i+1/2})\frac{C_{i+1}(t)-C_{i}(t)}{dx}+O(dx^2).
 \end{equation}
 
Using the mean concentration instead of the pointwise concentration (a second order approximation in $dx$),
\begin{equation}
C_{i}(t)=\frac{1}{dx}\int_{x_{i-1/2}}^{x_{i+1/2}}C(x,t)\,dx=\frac{1}{dx}M_{i}(t) + O(dx^2),
\end{equation}
we may write the approximate diffusion flux as,
 \begin{equation}
 J(x_{i-1/2}, t)=-D(x_{i-1/2})\frac{M_{i}(t)-M_{i-1}(t)}{dx^2} + O(dx^2),
 \end{equation}
 where the first order $O(dx)$ terms cancel due to the symmetry of the differences (midpoint), resulting in a second order accurate scheme. 
Therefore, the evolution of the mass in this approximate system is given by
\begin{equation}
\frac{d M_{i}(t)}{dt}= \frac{D(x_{i+1/2})M_{i+1}(t)-(D(x_{i+1/2})+D(x_{i-1/2}))M_{i}(t) + D(x_{i-1/2})M_{i-1}(t)}{dx^2},
\end{equation}
which is valid in the interior of the domain $i=1,...,n-1$.


Numerically, these boundary conditions can be enforced with the following approximations:
\begin{enumerate}
\item[(i')] The no flow condition can be imposed by simply considering either $J(x_{1/2},t)=0$ or $J(x_{n-1/2},t)=0$, so that
\begin{equation}
\frac{d M_{1}(t)}{dt}= D(x_{3/2})\frac{M_{2}(t)-M_{1}(t) }{dx^2},
\end{equation}
or
\begin{equation}
\frac{d M_{n-1}(t)}{dt}= D(x_{n-3/2})\frac{M_{n-2}(t)-M_{n-1}(t) }{dx^2}.
\end{equation}

\item[(ii')] 
At the interface points, we need to impose the concentration proportion ratio. This also implies in a mass proportionality, that may be written, for $x_n$, as
\begin{equation}
M_{n}=K M'_{n},
\label{eq:gap}
\end{equation}
where $M'_{n}$ is the mass on the neighbor compartment at the same $x_n$ point where the boundary of the reference compartment.

The flux condition can be discretized with finite differences as
\begin{equation}
 D(x_{n-1/2})\frac{M_{n}(t)-M_{n-1}(t) }{dx}= D(x_{n+1/2})\frac{M'_{n+1}(t)-M'_{n}(t) }{dx}.
 \label{eq:discrete_flux_interface}
\end{equation}

We may therefore enforce the gap \eqref{eq:gap} condition to hold in equation \eqref{eq:discrete_flux_interface} that leads to explicit expression of $M_{n}$ in terms of $M_{n-1}(t)$ and $M'_{n+1}(t)$:
\begin{equation}
M_{n}(t) = \frac{ KD(x_{n+1/2})}{D(x_{n+1/2})+KD(x_{n-1/2})} M'_{n+1}(t)+\frac{ KD(x_{n-1/2})}{D(x_{n+1/2})+KD(x_{n-1/2})} M_{n-1}(t).
\label{eq:gap+discrete_flux_interface}
\end{equation}
Analogously, we could isolate $M'_n$, to obtain, 
\begin{equation}
M'_{n}(t) = \frac{ D(x_{n+1/2})}{D(x_{n+1/2})+KD(x_{n-1/2})} M'_{n+1}(t)+\frac{ D(x_{n-1/2})}{D(x_{n+1/2})+KD(x_{n-1/2})} M_{n-1}(t).
\end{equation}


An analogous construction may be built when the interface is at point $x_0$.
\end{enumerate}



If the diffusion coefficient is known for all points defining the device, we now can now build a differential system only in time to be solved. For example, consider a compartment with no flow condition at $x_0$, interface at $x_n$ and constant diffusion coefficient $D$.

\begin{eqnarray*}
\frac{d M_{1}}{dt}&=& \frac{DM_{2}-DM_{1} }{dx^2},\\
\frac{d M_{i}}{dt}&=& \frac{DM_{i-1}-2D M_{i} + DM_{i+1}}{dx^2}, \,\, i=2,...,n-2,\\
\frac{d M_{n-1}}{dt}&=& \frac{1}{dx^2}\left(D M_{n-2}-D\frac{DK+2D'}{DK+D'} M_{n-1}  + D\frac{D'K}{KD+D'}M'_{n+1}\right),\\
\frac{d M'_{n+1}}{dt}&=& \frac{1}{dx^2}\left(D'\frac{D}{KD+D'}M_{n-1} -D'\frac{D'+2DK}{DK+D'} M'_{n+1}(t)  + D' M'_{n+2}(t)\right),\label{eq:system}
\end{eqnarray*}
Note that this system is coupled, in two places, with the system of the neighbor compartment: via diffusion coefficient $D'$; and via the mass at first inner point of the neighbor compartment $M'_{n+1}$.

\subsection{Multiples compartments}

If we have $N$ compartments, each with constant diffusion coefficient $D_l$, $l=1,2,...,N$, all with partition interfaces between, with constants $K_l$, $l=1,2,...N-1$, and no flow conditions at the outer boundaries, assuming that have discretized each compartment with $n_l$ sub-intervals, then we may write the fully coupled system as
\begin{equation}
\frac{d \vec{M}(t)}{dt}=A\vec{M}(t),
\end{equation}
where $A$ is a square matrix with dimension $n\times n$, where $n=\sum_{l=1}^{N}(n_l-1)$ and represents the interaction between all interior mass points (interface points are excluded and calculated diagnostically from equation \eqref{eq:gap+discrete_flux_interface}), given by equation \eqref{eq:system}. This is tridiagonal linear differential system with unknowns $\vec{M}(t)=[M_{1}(t), M_{2}(t), ..., M_{n_1-1}(t), M_{n_1+1}(t), ..., M_{n_1+n_2-1}(t), M_{n_1+n_2-1}(t), ..., M_{n}(t)]$, containing $n$ values. 


The numerical solution of the time evolution maybe obtained also via finite differences formulas. Given a time span we wish to simulate, say $[0,T]$, we may partition this interval into $N_t$ time steps, $t_m=m\, dt$, where $dt=T/N_t$, for $m=0,1,...,N_t$. Let $\vec{M}^m\approx  \vec{M}(t_m)$ be an approximate solution of $\vec{M}(t_m)$ at time $t_m$. An explicit solution of the problem, with first order accuracy with respect to time-step-size ($dt$), is
\begin{equation}
\vec{M}^{m+1}=\vec{M}^m+dt \,A\vec{M}^m
\end{equation}
which is the well-known explicit Euler scheme. While this scheme is attractive and easy to implement, it imposes serious stability conditions on the time-step-size, 
\begin{equation}
D\frac{dt}{dx^2}<\frac{1}{2},
\end{equation} 
where $D$ is the maximum diffusion coefficient in all the compartments. For small $dx$, the required $dt$ for the method to work (stably) is usually prohibitive. Therefore, implicit schemes are usually adopted. Here we will a Crank-Nicholson scheme, which is second order accurate with respect to $dt$ and unconditionally stable, may be written as
\begin{equation}
\vec{M}^{m+1}=\vec{M}^m+\frac{dt}{2}A\left(\vec{M}^m+\,\vec{M}^{m+1}\right).
\end{equation}

In this implicit scheme we cannot obtain the approximation for time $t_{k+1}$ directly from values at time $t_k$, as we need to solve a linear system of equation at each time-step. The system is defined as follows
\begin{equation}
(I-\frac{dt}{2}A)\vec{M}^{m+1}=(I+\frac{dt}{2}A)\vec{M}^{m},
\end{equation}
where $I$ is the $n\times n$ identity matrix. Since $A$ is tridiagonal, this system is very simple to solve with Gaussian Elimination.

It is convenient to work the theory in terms of mass, due to the conservation properties of the system. However, since the system is linear, assuming concentrations are piece-wise constant and the grid is uniform ($dx$ is the same for each compartment), then one may multiply all equations by $dx$ and obtain the same continuous equation for the concentrations,
\begin{equation}
\frac{d \vec{C}(t)}{dt}=A\vec{C}(t),
\end{equation}
where $\vec{C}(t)=[C_{1/2}(t), C_{3/2}(t), C_{5/2}(t), ..., C_{n-1/2}(t)]$. Therefore the same numerical method hold for the concentrations.

\section{Code: mextractmodel}

The numerical implementation, available via \url{https://github.com/pedrospeixoto/mextractmodel}, is divided into 3 main files:
\begin{itemize}
\item  \textbf{model.py} : Execution file (run with python model.py)
\item \textbf{mex-params.py} : Model parameters, to be manually changed by the user
\item \textbf{mextractmodel.py} : Main library for the microextraction model, to be imported where desired (for example in model.py)
\end{itemize}


%% The Appendices part is started with the command \appendix;
%% appendix sections are then done as normal sections
%% \appendix

%% \section{}
%% \label{}

%% For citations use: 
%%       \citet{<label>} ==> Jones et al. [21]
%%       \citep{<label>} ==> [21]
%%

%% If you have bibdatabase file and want bibtex to generate the
%% bibitems, please use
%%
%%  \bibliographystyle{elsarticle-num-names} 
%%  \bibliography{<your bibdatabase>}

%% else use the following coding to input the bibitems directly in the
%% TeX file.
\bibliographystyle{elsarticle-num-names} 
\bibliography{references.bib}

%\begin{thebibliography}{00}

%% \bibitem[Author(year)]{label}
%% Text of bibliographic item



%\end{thebibliography}

\end{document}

\endinput
%%
%% End of file `elsarticle-template-num-names.tex'.
